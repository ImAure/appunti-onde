\chapter{L'Equazione delle Onde}\label{ch:2}
[BOZZETTI]
\subsection{Onde piane monocromatiche}
        Le funzioni coseno che compaiono nel risultato precedente appartengono a una classe di funzioni particolari che meritano qualche parola in pi\`u. Chiamiamo le funzioni $A\cos\qty(qx\pm\omega t+\phi)$ \emph{onde piane monocromatiche}, le quali vengono dette \emph{progressive} nel caso del segno $-$ e \emph{regressive} nel caso del segno $+$. Introduciamo anche della nomenclatura utile\cite{Focardi2014-wy}:
        \begin{itemize}
            \item $q$ \`e il modulo del \emph{vettore d'onda};
            \item $\omega$ \`e la \emph{frequenza angolare} o \emph{pulsazione} dell'onda;
            \item $A$ \`e l'\emph{ampiezza} dell'onda;
            \item $qx\pm\omega t+\phi$ \`e la \emph{fase} dell'onda.
        \end{itemize}
        Definiamo inoltre delle quantit\`a utili al resto della trattazione:
        \begin{itemize}
            \item $\displaystyle T=\frac{2\pi}{\omega}$ \`e detto \emph{periodo} e rappresenta il tempo necessario a un punto fissato nello spazio affinch\'e si ripeta la stessa fase;
            \item $\displaystyle \lambda=\frac{2\pi}{q}$ \`e la \emph{lunghezza d'onda} e, fissato il tempo, rappresenta la distanza tra due punti con la stessa fase;
            \item $\displaystyle v=\frac{\omega}{q}=\frac{\lambda}{T}$ \`e detta \emph{velocit\`a di fase} o \emph{di propagazione} e rappresenta la velocit\`a con cui un punto a fase fissata si sposta nello spazio.
        \end{itemize}
        \par Alla luce di quanto detto possiamo verificare facilmente che nel caso della catena di molle le lunghezze d'onda possono assumere solo ben determinati valori, dalla \eqref{eq:q}: $$\lambda_q=\frac{2\pi}{q}=2\pi\frac{Na}{2m\pi}=\frac{L}{m},$$ che si vedono essere frazioni di $L$ al variare di $m\in\Z$, $m>0$. Per il caso $m=0$, a cui sappiamo corrispondere il moto di traslazione, il $\displaystyle\lim_{q\to 0^+}\frac{2\pi}{q}=+\infty$ suggerisce proprio un moto che si ripete a distanza infinita.