\chapter{Cenni su Numeri e Funzioni in $\C$}\label{app:A}
\section{Rappresentazione dei numeri complessi}
    I numeri complessi sono un'utile estensione dei numeri reali che fanno uso dell'unit\`a immaginaria $i$ definita come $i^2=-1$. L'insieme $\C$ dei numeri complessi \`e isomorfo a $\R^2$ e a ogni coppia $(x,y)$ si pu\`o associare il numero $z=x+iy$, espresso nella forma algebrica. $x=\Re\qty(z)$ e $y=\Im\qty(y)$ sono detti rispettivamente \emph{parte reale} e \emph{parte immaginaria}.
    \subsection{Forma trigonometrica e forma esponenziale}
        Rappresentando il numero $z=x+iy\neq 0$ nel piano di Argand-Gauss, possiamo osservare che la sua posizione \`e univocamente determinata conoscendo due parametri, il \emph{modulo} $\rho=\sqrt{z\overline{z}}=\sqrt{x^2+y^2}$ e l'\emph{argomento} $\displaystyle\theta=\arctan{\frac{y}{x}}$. \`E quindi possibile eseguire la trasformazione:
        \begin{equation*}
        \begin{cases}
            x=\rho\cos\theta\\
            y=\rho\sin\theta
        \end{cases}
        \end{equation*}
        da cui $z=\rho\qty(\cos\theta+i\sin\theta)$, detta \emph{forma trigonometrica}. \par Adesso consideriamo il numero $\rho e^{i\theta}$ e scriviamolo attraverso la serie esponenziale: 
        \begin{multline*}
            \rho e^{i\theta}=\rho\sum_{n=0}^{+\infty}\frac{\qty(i\theta)^n}{n!}=\rho\qty[\sum_{n=0}^{+\infty}\frac{\qty(i\theta)^{2n}}{\qty(2n)!}+\sum_{n=0}^{+\infty}\frac{\qty(i\theta)^{2n+1}}{\qty(2n+1)!}]=\\ =\rho\qty[\sum_{n=0}^{+\infty}\qty(-1)^n\frac{\theta^{2n}}{\qty(2n)!}+i\sum_{n=0}^{+\infty}\qty(-1)^n\frac{\theta^{2n+1}}{\qty(2n+1)!}]=\rho\qty(\cos\theta+i\sin\theta)
        \end{multline*}    
        dove si \`e usato $i^{2n}=\qty(-1)^n$ e $i^{2n+1}=i\qty(-1)^n$. Osserviamo dunque che esiste l'uguaglianza tra la forma trigonometrica e quella che chiamiamo \emph{forma esponenziale}; la relazione che le lega \`e detta \emph{Identit\`a di Eulero}.
    \subsection{Formule di Eulero}
        Ricaviamo ora due relazioni utili partendo dall'Identit\`a appena otteuta. Consideriamo $z=\rho e^{i\theta}=\rho\qty(\cos\theta+i\sin\theta)$ e il suo coniugato $\overline{z}=\rho e^{-i\theta}=\rho\qty(\cos\theta-i\sin\theta)$.
        Da semplici conti:
        \begin{align*}
            z+\overline{z}=\rho e^{i\theta}+\rho e^{-i\theta}&=2\rho\cos\theta \iff \cos\theta=\frac{e^{i\theta}+e^{-i\theta}}{2}\\
            z-\overline{z}=\rho e^{i\theta}-\rho e^{-i\theta}&=2\rho i\sin\theta \iff \sin\theta=\frac{e^{i\theta}-e^{-i\theta}}{2i}\\
        \end{align*}
\section{Operazioni coi numeri complessi}
Vediamo brevemente come eseguire facilmente le operazioni algebriche pi\`u importanti in $\C$.
    \subsection{Somma e prodotto}
        Per sommare e sottrare due numeri complessi \`e utile fare riferimento alla loro forma algebrica, infatti la loro somma si esegue per componenti, esattamente come per i vettori di $\R^2$. $$z_1=x_1+iy_1\ ,\ z_2=x_2+iy_2$$ $$z_1\pm z_2=\qty(x_1\pm x_2)+i\qty(y_1\pm y_2)$$ Il prodotto si pu\`o eseguire come prodotto di binomi, ricordando $i^2=-1$: $$z_1z_2=\qty(x_1+iy_1)\qty(x_2+iy_2)=x_1x_2-y_1y_2+i\qty(x_1y_2+x_2y_1)$$
        Tuttavia, pu\`o essere pi\`u utile ricondursi alla forma esponenziale, in quanto, detti $z_1=\rho_1e^{i\theta}$, $z_2=\rho_2e^{i\phi}$, si ha: $$z_1z_2=\rho_1e^{i\theta}\rho_2e^{i\phi}=\rho_1\rho_2e^{i\qty(\theta+\phi)}$$ $$\frac{z_1}{z_2}=\rho_1e^{i\theta}\frac{1}{\rho_2}e^{-i\phi}=\frac{\rho_1}{\rho_2}e^{i\qty(\theta-\phi)}$$ Da cui segue subito l'elevamento a potenza: $$z^r=\qty(\rho e^{i\theta})^r=\rho^re^{ir\theta}$$