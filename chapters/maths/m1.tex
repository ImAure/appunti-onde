\chapter{Numeri Complessi}\label{ap:A}
I numeri complessi sono la naturale estensione del campo reale $\qty(\R,+,\cdot)$ che, pur sacrificando la relazione d'ordine, permettono di avere una struttura algebricamente chiusa. \par I contenuti di questo capitolo fanno prevalentemente riferimento alle fonti \cite{Di_Fazio2013-vk,Presilla2013-uz}, con alcune modifiche e aggiunte personali.
\section{Costruzione di $\C$}
Sia $\R^2=\qty{\qty(x,y):x,y\in\R}$, poniamo $z_1=\qty(x_1,\ y_1)$, $z_2=\qty(x_2,\ y_2)$ e siano definite le operazioni:
\begin{align*}
    \func{+}{\R^2\times\R^2}{\R^2}\quad&\qty(z_1,\ z_2)\mapsto\qty(x_1+x_2,\ y_1+y_2)\\
    \func{\cdot}{\R^2\times\R^2}{\R^2}\quad&\qty(z_1,\ z_2)\mapsto\qty(x_1x_2-y_1y_2,\ x_1y_2+x_2y_1)
\end{align*}
Chiamiamo $\C=\qty(\R^2,+,\cdot)$ la struttura cos\`i costruita, che si dimostra facilmente essere un campo.
\subsection{Forme e propriet\`a dei numeri complessi}
    Definiamo \emph{modulo} di un numero complesso $z$ la quantit\`a: $$\abs{z}=\sqrt{x^2+y^2}\in\R.$$ 
    \subsubsection{Forma algebrica}
        Osserviamo che l'insieme $\qty{\qty(x,y)\in\R^2:y=0}$ \`e isomorfo a $\R$, per cui poniamo $x=\qty(x,0)$. Sia inoltre $i=\qty(0,\ 1)$, da cui:
        \begin{align*}
            &y\cdot i=\qty(y,0)\cdot(0,1)=(0,y)\\
            &i^2=\qty(0,1)\cdot\qty(0,1)=(-1,0)=-1
        \end{align*}
        Facendo la posizione $iy=\qty(0,y)$, ogni numero complesso $z=\qty(x,y)$ pu\`o essere scritto nella sua \emph{forma algebrica} $z=x+iy$, infatti: $$x+iy=\qty(x,0)+\qty(0,y)=(x,y)$$
        Dato un numero complezzo $z=x+iy$, definiamo:
        \begin{itemize}
            \item $\Re z=x$, detta \emph{parte reale} di $z$;
            \item $\Im z=iy$, detta \emph{parte immaginaria} di $z$;
            \item $\overline{z}=x-iy$, detto \emph{complesso coniugato} di $z$.
        \end{itemize}
        Vale inoltre la propriet\`a $\abs{z}=z\overline{z}$.
    \subsubsection{Forma trigonometrica ed esponenziale}
        Come noto, \`e possibile rappresentare i punti del piano $\qty(x,y)\in\R^2$, escluso $\qty(0,0)$ in coordinate polari, fissando le due coordinate $\rho$ e $\theta$, detti rispettivamente \emph{raggio} e \emph{angolo polare}. \par Pensando ai numeri complessi come punti del piano, si osserva: $$x=\rho\cos\theta,\quad y=\rho\sin\theta,$$ da cui: $$z=\rho\qty(\cos\theta+i\sin\theta),$$ che viene detta \emph{forma trigonometrica} di $z$. Segue naturalmente che:
        \begin{itemize}
            \item $\rho=\abs{z}$;
            \item $\theta$ \`e l'angolo che risolve $x=\rho\cos\theta$ e $y=\rho\sin\theta$;
            \item $\overline{z}=\rho\qty(\cos\theta-i\sin\theta)=\rho\qty[\cos\qty(-\theta)+i\sin\qty(-\theta)]$.
        \end{itemize}
        \par Consideriamo ora la quantit\`a $e^{i\theta}$ e dimostriamo che rappresenta ancora un numero complesso. \`E possibile interpretare l'esponenziale come una somma infinita di termini, esprimendo la funzione $e^z$ come serie di Taylor centrata in $0$: $$e^{i\theta}=\sum_{n=0}^{+\infty}\frac{\qty(i\theta)^n}{n!}=\sum_{n=0}^{+\infty}i^n\frac{\theta^n}{n!}=\sum_{n=0}^{+\infty}\qty(-1)^n\frac{\theta^{2n}}{\qty(2n)!}+i\sum_{n=0}^{+\infty}\qty(-1)^n\frac{\theta^{2n+1}}{\qty(2n+1)!}$$ dove \`e evidente la presenza degli sviluppi di seno e coseno, da cui: 
        \begin{equation}
            e^{i\theta}=\cos\theta+i\sin\theta,
            \label{eq:euler:1}
        \end{equation}
        la quale prende il nome di \emph{formula di Eulero}.\footnote{Da cui la pi\`u popolare \emph{identit\`a di Eulero}: $e^{i\pi}+1=0$} Segue che qualunque numero complesso $z\neq 0$ pu\`o essere espresso come: $$z=\rho\qty(\cos\theta+i\sin\theta)=\rho e^{i\theta},$$ che viene detta \emph{forma esponenziale} del numero complesso.
        \par Dalla formula di Eulero seguono altre importanti relazioni. Consideriamo ancora $z=e^{i\theta}$ e il suo coniugato, che per le precedenti \`e naturalmente $\overline{z}=e^{-i\theta}$:
        \begin{align}
            z+\overline{z}=e^{i\theta}+e^{-i\theta}=2\cos\theta\ \iff\ \cos\theta=\frac{e^{i\theta}+e^{-i\theta}}{2}\\
            z-\overline{z}=e^{i\theta}-e^{-i\theta}=2i\sin\theta\ \iff\ \sin\theta=\frac{e^{i\theta}-e^{-i\theta}}{2i}
        \end{align}
        Inoltre, per qualunque $z_1,z_2\in\C$ si ha:
            $$z_1z_2=\rho_1e^{i\theta_1}\rho_2e^{i\theta_2}=\rho_1\rho_2e^{i\qty(\theta_1+\theta_2)}.$$
        da cui:
        \begin{equation}
            \overline{z_1z_2}=\rho_1\rho_2\myexp{-i\qty(\theta_1+\theta_2)}=\rho_1\myexp{-i\theta_1}\rho_2\myexp{-i\theta_2}=\overline{z}_1\overline{z}_2
            \label{eq:prodottoconiugati}
        \end{equation}
    \section{Formula di De Moivre e radici in $\C$}
        Per poter operare al meglio coi numeri complessi, introduciamo dei metodi per eseguire con facilit\`a le operazioni di elevamento a potenza ed estrazione di radice. Consideriamo due numeri $z_1=\rho_1e^{i\theta_1}$ e $z_2=\rho_2e^{i\theta_2}$ e il loro prodotto:
            $$z_1z_2=\rho_1\rho_2e^{i\qty(\theta_1+\theta_2)}.$$
        Da questa scrittura possiamo inutire che per qualunque $z\neq 0$ e $k\in\Z$, possiamo scrivere:
        \begin{equation}
            z^k=\rho^ke^{ik\theta}=\rho^k\qty(\cos k\theta+i\sin k\theta)
            \label{eq:demoivre}
        \end{equation}
        che viene detta formula di De Moivre e pu\`o essere dimostrata per induzione su $k$.
        \par Il lettore potrebbe essere indotto a pensare che una scrittura simile sia corretta anche per gli esponenti razionali, come nel caso dell'estrazione di radice. Tuttavia \`e necessario tenere a mente che estrarre la radice $n$-esima, con $n\in\N$, di un numero complesso $a$ vuol dire risolvere l'equazione:
            $$z^n=a,$$
        che in $\C$ ha $n$ soluzioni distinte, in virt\`u del teorema fondamentale dell'algebra. Per tali soluzioni \`e ancora una volta possibile fare riferimento alla forma esponenziale e trigonometrica dei numeri complessi, infatti ricordando che per la periodicit\`a di seno e coseno $a=\rho\myexp{i\qty(\theta+2m\pi)}$, $m\in\Z$ deve essere:
            $$z^n=\rho\myexp{i\qty(\theta+2m\pi)}\ \iff\ z=\rho^{\frac{1}{n}}\myexp{i\qty(\frac{\theta+2m\pi}{n})},\quad m=0,\dots,n-1.$$